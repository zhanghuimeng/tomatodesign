% !Mode:: "TeX:UTF-8"
\documentclass{article}
\usepackage[hyperref, UTF8]{ctex}
\usepackage[dvipsnames]{xcolor}
\usepackage{geometry}
\usepackage{amsmath}
\usepackage{amsfonts}
\usepackage{listings}
\usepackage{pgfplotstable}
\usepackage{pgfplots}
\usepackage{fontspec}
\usepackage{booktabs} % 表格上的不同横线
%\setmainfont{Consolas} %设置英文字体

\definecolor{mygreen}{rgb}{0,0.6,0}
\definecolor{mygray}{rgb}{0.5,0.5,0.5}
\definecolor{mymauve}{rgb}{0.58,0,0.82}
\lstset{ %
    backgroundcolor=\color{white},   % choose the background color
    basicstyle=\footnotesize\ttfamily,        % size of fonts used for the code
    columns=fullflexible,
    breaklines=true,                 % automatic line breaking only at whitespace
    captionpos=b,                    % sets the caption-position to bottom
    tabsize=4,
    backgroundcolor=\color[RGB]{245,245,244},            % 设定背景颜色
    commentstyle=\color{mygreen},    % comment style
    escapeinside={\%*}{*)},          % if you want to add LaTeX within your code
    keywordstyle=\color{blue},       % keyword style
    stringstyle=\color{mymauve}\ttfamily,     % string literal style
    showstringspaces=false,                % 不显示字符串中的空格
    frame=none,
    rulesepcolor=\color{red!20!green!20!blue!20},
    % identifierstyle=\color{red},
    language=c++,
}

% 设置hyperlink的颜色
\newcommand\myshade{85}
\colorlet{mylinkcolor}{violet}
\colorlet{mycitecolor}{YellowOrange}
\colorlet{myurlcolor}{Aquamarine}


\title{软件工程大作业文档}
\author{Tomato}
\date{2018年1月}

\begin{document}

	\maketitle
	\section{系统/子系统设计(结构设计)说明(后台部分)}
		\subsection{引言}
			\subsubsection{系统概述}
				后台分为两个部分,上层是Controller,是用来处理业务逻辑的,下层是数据库,给上层的Controller提供相应的服务。下面会分两块来具体介绍这两部分。详见\ref{接口设计说明}和\ref{数据库(顶层)设计说明}。

			\subsubsection{文档概述}
				本文档将围绕Spring Boot架构来介绍后台部分。
		\subsection{系统级设计决策}
			本系统后台主要使用Spring Boot架构,它支持在开启服务的时候,接受HttpRequest或websocket并进行后台处理,然后返回相应的结果。系统接受的输入是HttpRequest或websocket,详见\ref{接口设计说明},对于每一个输入的响应,被Controller定义,不同的request会有不同的Controller对其进行处理并返回相应的结果。系统处理的过程中,依赖于两部分的内容:用户的request中的具体内容和底层数据库已有的信息。详见\ref{数据库(顶层{设计说明}}。该系统的数据库存在服务器上,服务器的管理人员可以通过database asdan来访问数据库并查看当前数据库中的所有信息。该系统对服务器的要求是要求服务器能通过maven运行项目,并配有mysql数据库。
		\subsection{系统体系结构设计}
			\subsubsection{系统总体设计}
				\paragraph{概述}
					本系统要实现的功能详见《prj9ASDAN模拟商业竞拍交易大赛系统》。本系统要实现的性能:响应较快、支持并发、有一定的安全性、可在各个平台上运行、可移植等。本系统支持在Windows、Unix、Linux环境下运行。
				\paragraph{设计思想}
					本系统顶层核心要处理的两个问题是对于HttpRequest的处理和对于websocket的处理,而底层核心要处理的问题是数据库的维护等问题。如果直接从头开始开发一套框架,则要实现的内容可能比我们当前写的多得多。所以我们后台根据这个需求找到了Spring Boot框架。Spring Boot 很好地通过了Annotations来实现我们需要的对于前端web端的响应,也实现了我们需要的和数据库的连接、并发的响应等等。本系统并不需要很高深的算法,但是对逻辑上的要求十分的多和细致。
				\paragraph{基本处理流程}
                    下面以用户登录为例来讲解后台对于HttpRequest的处理流程。

                    \includegraphics[scale = .3]{fig/基本数据.png}

                    当前端发来HttpRequest或websocket的时候,Controller首先会提取出发来的数据内容,然后将根据业务逻辑处理数据,在处理数据的过程中,可能需要访问数据库中已有的数据,如在这个情况下需要对拍数据库中的账号和密码。也有可能需要修改后台数据库,但是图中的例子不需要。在处理完数据之后,Controller会把数据重新包装,然后根据API文档的说明传回前端。整个数据处理的过程可以理解为从前端发来数据,然后Controller向数据库query数据,然后处理完数据之后又save到数据库中,最后返回给前端。
                \paragraph{系统体系结构}
                    \includegraphics[scale = .3]{fig/架构图.png}

                    如图,前端的功能是响应用户的操作,并把用户操作的数据根据API文档中的要求进行包装,作为HttpRequest或websocket发往后端,后端Controller对于包装好的数据解码,并query Service,和数据库交互,完成数据的处理,并完成数据库的更新。最后将HttpRequest返回给前端或将websocket转发。
            \subsubsection{接口设计}
                \includegraphics[scale = .3]{fig/架构图.png}
		\subsection{系统出错处理设计}
			系统出错后,后台数据库的内容不会被清空,并且我们是在处理request流程中对数据库是实时更新的,宕机之后数据库的内容还是会保留,当下次启动服务器的时候它还能从原数据库中获得原来的数据,并根据这些数据进行处理。

			当系统出错后,建议重启服务器。
		\subsection{尚待解决的问题}
			微信后端的websocket的测试问题。
		\subsection{需求的可追踪性}
		\subsection{注解}

	\section{接口设计说明}
		\label{接口设计说明}
		\subsection{引言}
			\subsubsection{系统概述}
				接口主要分为两大类的接口,一类是Admin的接口,另一类是Client的接口。Client有两种接口,一种是web端的接口,另一类是微信端的接口。Admin本身是一个package,两种Client公用一个package,名为client。Web端的接口包括了正常HttpRequest和Web端的Websocket。Web端的Websocket使用StompClient协议,在后端使用Spring Boot自带的Controller,即@MessageMapping和@SendTo来完成对socket内容的转发。而微信端在HttpRequest上和Web端共用一个API,并使用裸的socket来回复websocket。
			\subsubsection{文档概述}
				文档首先介绍了使用的文献,然后对于每个特定的接口进行了具体的说明。
		\subsection{引用文件}
			本文档引用了我们在开发过程中撰写的《RestAPI》文档,详情可查看\href{https://github.com/wenj/tomatodesign/blob/master/REST%20API.tex}{RestAPI.tex}。

		\subsection{Admin接口}

\subsubsection{Login Admin}
管理员登录。

\paragraph*{Request}
\begin{lstlisting}
POST /api/admin/login

Host: localhost:8080
Auth:
Content-type: application/json
Accept: application/json

{
    "username": "admin",
    "password": "admin",
}
\end{lstlisting}

\paragraph*{Returns}
\begin{lstlisting}
HTTP 200 OK
{
    "username": "admin",
    "token": "1283091828021803120",
}

\end{lstlisting}

\paragraph*{Error}
\begin{lstlisting}
HTTP 401 NOTAUTHORIZED
{
    "error": "Admin with username admin doesn\'t exist or password is wrong."
}
\end{lstlisting}

\subsubsection{Change Admin Password}
更改管理员密码。

\paragraph*{Request}
\begin{lstlisting}
POST /api/admin/update

Host: localhost:8080
Auth:
Content-type: application/json
Accept: application/json

{
    "username": "admin",
    "prePassword": "admin",
    "newPassword": "newpwd",
}
\end{lstlisting}

\paragraph*{Returns}
\begin{lstlisting}
HTTP 200 OK

\end{lstlisting}

\paragraph*{Error}
\begin{lstlisting}
HTTP 401 NOTAUTHORIZED
{
    "error": "Wrong password of admin."
}
\end{lstlisting}


\subsubsection{Get All Competitions}

列出全部比赛。

Status是"not\_start", "auction\_not\_record", "auction\_recorded", "trade", "rest", "end"之一。

\paragraph*{Request}
\begin{lstlisting}
GET /api/admin/competition/getall

Host: localhost:8080
Auth:
Content-type: application/json
Accept: application/json
\end{lstlisting}

\paragraph*{Returns}
\begin{lstlisting}
HTTP 200 OK

[
    {
        "id": "competiton1_id",
        "username": "competition1",
        "status": "auction"
    },
    {
        "id": "competiton2_id",
        "username": "competition2",
        "status": "end"
    }
]
\end{lstlisting}

\paragraph*{Error}
\begin{lstlisting}
HTTP 204 NO CONTENT
\end{lstlisting}

\subsubsection{Create Competition}
新建一场比赛。注意,底层也要生成机器的id。注意每场比赛的基本配置(比赛名称,参赛人数)只能创建一次,不能修改。

\paragraph*{Request}
\begin{lstlisting}
POST /api/admin/competition/new

Host: localhost:8080
Auth:
Content-type: application/json
Accept: application/json

{
    "username": "competition_username",
    "round": 2,
    "startWealth": 1000,

    "teamNum": 2,
    "participantNum": 3,
    "team":
    [
        {
            "username": "team1",
            "participant": ["mem11", "mem12", "mem13"],
            "password": "111111",
        },
        {
            "username": "team2",
            "participant": ["mem21", "mem22", "mem23"],
            "password": "222222",
        }
    ]

    "roundParameter":
    [
        {
            "machineStartPrice": [300, 350, 400],
            "machineNum": [1, 1, 1],
            "materialProduceCost": [10, 20, 30],
            "time": 900,
        },
        {
            "machineStartPrice": [300, 350, 400],
            "machineNum": [1, 1, 1],
            "materialProduceCost": [10, 20, 30],
            "time": 900,
        }
    ]
}
\end{lstlisting}

\paragraph*{Returns}
\begin{lstlisting}
HTTP 201 CREATED

\end{lstlisting}

\paragraph*{Error}
\begin{lstlisting}
HTTP 404 NOT FOUND
{
    "error":"Unable to delete. Competition with id xxx not found."
}
\end{lstlisting}


\subsubsection{Delete Competition By ID}

通过ID删除比赛。

\paragraph*{Request}
\begin{lstlisting}
DELETE /api/admin/competiton/id={competition_id}

Host: localhost:8080
Auth:
Content-type: application/json
Accept: application/json
\end{lstlisting}

\paragraph*{Returns}
\begin{lstlisting}
HTTP 200 OK
[
    {
        "id": "competiton2_id",
        "username": "competition2",
        "status": "end"
    }
]
\end{lstlisting}

\paragraph*{Error}
\begin{lstlisting}
{
    "error":"Unable to delete. Competition with id xxx not found."
}
\end{lstlisting}

\subsubsection{Update Competition Status}

需要进入下一环节时,管理员端会向服务器发送更新比赛状态的请求,服务器返回当前比赛信息以便管理员端更新到最新的比赛状态。

\paragraph*{Web Socket}
\begin{lstlisting}
MessageMapping: /api/admin/status/update/id=3

SendTo: /api/admin/status/id=3

EndPoint:  http://127.0.0.1:8090/competitionStatus

Send JSON Pattern:
{
    "status": "auction"("not_start", "auction_not_record", "auction_recorded", "trade", "rest", "end")
    "round": 0/1/2/3
    "timeLeft":227(s)
 }

Get JSON Pattern:
{
    "status": "auction"("not_start", "auction_not_record", "auction_recorded", "trade", "rest", "end")
    "round": 0/1/2/3
}
\end{lstlisting}

\paragraph*{Error}
\begin{lstlisting}
HTTP 404 NOT FOUND
{
    "error":"Unable to update. Competition with id xxx not found"
}
\end{lstlisting}

\subsubsection{Get Auction Machine}

获得某场比赛某一轮拍卖机器的初始信息。

\paragraph*{Request}
\begin{lstlisting}
GET /api/admin/competition/auction/id={id}/round={round}

Host: localhost:8080
Auth:
Content-type: application/json
Accept: application/json
\end{lstlisting}

\paragraph*{Returns}
\begin{lstlisting}
HTTP 200 OK
[
    {
        "machineId": "machine1",
        "type": "Wood",
        "startPrice": 200,
    },
    {
        "machineId": "machine2",
        "type": "Brick",
        "startPrice": 300,
    },
    {
        "machineId": "machine3",
        "type": "Cement",
        "startPrice": 400,
    }
]
\end{lstlisting}

\paragraph*{Error}
\begin{lstlisting}
HTTP 404 NOT FOUND
{
    "error": "Competition with id xxx not found." (or Competition with id xxx does not have round xxx)
}
\end{lstlisting}

\subsubsection{Record Auction Result}

登记某场比赛某一轮的拍卖结果。

\paragraph*{Request}
\begin{lstlisting}
POST /api/admin/competition/record/id={id}/round={round}

Host: localhost:8080
Auth:
Content-type: application/json
Accept: application/json
\end{lstlisting}

\paragraph*{Returns}
\begin{lstlisting}
HTTP 200 OK
[
    {
        "machineId": "machine1",
        "teamId": "team1",
        "price": 2000,
    },
    {
        "machineId": "machine2",
        "teamId": "brick",
        "price": 3000,
    },
    {
        "machineId": "machine3",
        "teamId": "cement",
        "price": 4000,
    }
]
\end{lstlisting}

\paragraph*{Error}
\begin{lstlisting}
HTTP 404 NOT FOUND
{
    "error": "Competition with id xxx not found." (or Competition with id xxx does not have round xxx)
}
\end{lstlisting}


\subsubsection{Get Competition Property}

从服务器按id获取某一比赛的各种属性。如果该比赛的属性尚未被设置,则该项为空。属性包括名称、比赛轮数(如果比赛已开始,则不能删除已开始或结束的轮)、比赛各项参数(不能修改已开始或结束的轮的参数)、机器的id等等。

\paragraph*{Request}
\begin{lstlisting}
GET /api/admin/competition/property/id={id}

Host: localhost:8080
Auth:
Content-type: application/json
Accept: application/json
\end{lstlisting}

\paragraph*{Returns}
\begin{lstlisting}
HTTP 200 OK
{
    "id": "competition_id",
    "username": "competition_username",
    "status": "not started",
    "teamNum": 1,
    "participantNum": 2,
    "team":
    [
        {
            "username": "team1",
            "participant": ["member1", "member2", "member2"],
            "password": "password",
        }
    ]
    "round": 1,
    "startWealth": 1000,
    "roundParameter":
    [
        {
            "machineStartPrice": [300, 350, 400],
            "machineNum": [1, 1, 1],
            "materialProduceCost": [10, 20, 30],
            "time": 900,
        }
    ]
}

\end{lstlisting}

\paragraph*{Error}
\begin{lstlisting}
HTTP 404 NOT FOUND
{
    "error": "Competition with id xxx not found."
}
\end{lstlisting}

\subsubsection{Update Competition Property}

更新比赛的各种属性。属性包括名称、比赛轮数(如果比赛已开始,则不能更改)、比赛各项参数(不能修改已开始或结束的轮的参数)。

\paragraph*{Request}
\begin{lstlisting}
PUT /api/admin/competition/property/id={id}

Host: localhost:8080
Auth:
Content-type: application/json
Accept: application/json

{
    "round": 2,
    "startWealth": 1000,
    "round_parameter":
    [
        {
            "machineStartPrice": [300, 350, 400],
            "machineNum": [1, 1, 1],
            "materialProduceCost": [10, 20, 30],
            "time": 900,
        },
        {
            "machineStartPrice": [300, 350, 400],
            "machineNum": [1, 1, 1],
            "materialProduceCost": [10, 20, 30],
            "time": 900,
        }
    ]
}
\end{lstlisting}

\paragraph*{Returns}
\begin{lstlisting}
HTTP 201 CREATED
\end{lstlisting}

\paragraph*{Error}
\begin{lstlisting}
HTTP 404 NOT FOUND
{
    "error": "Competition with id xxx not found."
}
\end{lstlisting}

\begin{lstlisting}
HTTP 400 INVALID REQUEST
{
    "error": "Cannot update competition id xxx with given changes."
}
\end{lstlisting}

\subsubsection{Get Competition Information}
获取当前比赛信息,包括队伍的数量、资产、交易记录、机器的使用情况等。

\paragraph*{Request}
\begin{lstlisting}
GET /api/admin/competition/info/id={id}

Host: localhost:8080
Auth:
Content-type: application/json
Accept: application/json
\end{lstlisting}

\paragraph*{Returns}
\begin{lstlisting}
HTTP 200 OK
{
    "id": "competition_id",
    "username": "competition_username",
    "status": "not_start",
    "round": 2,
    "presentRound": 0,
    "teamInfo":
    [
        {
            "id": "id1",
            "wealth": 100,
            "material": [30, 40, 50],
            "machine":
            [
                 {
                    "id": "machine1_id",
                    "type": "type1",
                    "left": 3
                 },
                 {
                    "id": "machine2_id",
                    "type": "type2",
                    "left": 2
                 }
            ]
        },
        {
            "id": "id2",
            "wealth": 100,
            "material": [30, 40, 50],
            "machine":
            [
                 {
                    "id": "machine1_id",
                    "type": "type1",
                    "left": 3
                 },
                 {
                    "id": "machine2_id",
                    "type": "type2",
                    "left": 2
                 }
            ]
        }
    ],
    "trade_history":
    [
        {
            "time": "yyyy-MM-dd'T'HH:mm:ss.SSS'Z'",
            "sell": "team_id1",
            "buy": "team_id2",
            "content": {"wood": 1},
            "price": 10
        },
         {
            "time": "yyyy-MM-dd'T'HH:mm:ss.SSS'Z'",
            "sell": "team_id1",
            "buy": "team_id2",
            "content": {"machine_wood": 1},
            "price": 20
        }
    ]
}

\end{lstlisting}
content中是wood,brick,cement,machine\_wood,machine\_brick,machine\_cement中的一个。

\paragraph*{Error}
\begin{lstlisting}
HTTP 404 NOT FOUND
{
    "error": "Competition with id 1 not found."
}
\end{lstlisting}

\subsubsection{Record machine owner}
向服务器发送对比赛的更新信息。增加机器、分配财产之类的。

\paragraph*{Request}
\begin{lstlisting}
PUT /api/admin/competition/info/id={id}

Host: localhost:8080
Auth:
Content-type: application/json
Accept: application/json

{
    "round": 2,
    "present_round": 0,
    "team_info":
    [
        {
            "id": "id1",
            "wealth": "100",
            "machine":
            [
                 {
                    "id": "machine1_id",
                    "left": 3
                 },
                 {
                    "id": "machine2_id",
                    "left": 2
                 }
            ]
        },
        {
            "id": "id2",
            "wealth": "100",
            "machine":
            [
                 {
                    "id": "machine1_id",
                    "left": 3
                 },
                 {
                    "id": "machine2_id",
                    "left": 2
                 }
            ]
        }
    ]
}
\end{lstlisting}

\paragraph*{Returns}
\begin{lstlisting}
HTTP 200 OK
{
    "id": "competition_id",
    "username": "competition_username",
    "status": "not started",
    "round": "2",
    "present_round": "0",
    "team_info":
    [
        {
            "id": "id1",
            "wealth": 100,
            "material": [30, 40, 50],
            "machine":
            [
                 {
                    "id": "machine1_id",
                    "type": "type1",
                    "left": 3
                 },
                 {
                    "id": "machine2_id",
                    "type": "type2",
                    "left": 2
                 }
            ]
        },
        {
            "id": "id2",
            "wealth": 100,
            "material": [30, 40, 50],
            "machine":
            [
                 {
                    "id": "machine1_id",
                    "type": "type1",
                    "left": 3
                 },
                 {
                    "id": "machine2_id",
                    "type": "type2",
                    "left": 2
                 }
            ]
        }
    ],
    "trade_history":
    [
        {
            "time": "hh:MM:ss",
            "sell": "team_id1",
            "buy": "team_id2",
            "content": {"wood": 1},
            "price": 10
        },
         {
            "time": "hh:MM:ss",
            "sell": "team_id1",
            "buy": "team_id2",
            "content": {"machine_wood": 1},
            "price": 20
        }
    ]
}

\end{lstlisting}
content中是wood,brick,cement,machine\_wood,machine\_brick,machine\_cement中的一个。

\paragraph*{Error}
\begin{lstlisting}
HTTP 404 NOT FOUND
{
    "error": "Competition with id xxx not found."
}
\end{lstlisting}

\begin{lstlisting}
HTTP 400 INVALID REQUEST
{
    "error": "Cannot update competition id xxx with given information."
}
\end{lstlisting}

\subsection{Client端接口}
首先是HttpRequest的接口。

\subsubsection{Login Client}
用户登录。

\paragraph*{Request}
\begin{lstlisting}
POST /api/client/login

Host: localhost:8080
Auth:
Content-type: application/json
Accept: application/json

{
    "username": "client",
    "password": "client",
}
\end{lstlisting}

\paragraph*{Returns}
\begin{lstlisting}
HTTP 200 OK
{
    "username": "client",
     "id": "3",
    "token": "1283091828021803120",
}

\end{lstlisting}

\paragraph*{Error}
\begin{lstlisting}
HTTP 401 NOTAUTHORIZED
{
    "error": "Client with userusername admin doesn\'t exist or password is wrong."
}
\end{lstlisting}

\subsubsection{Get Information}
这个接口在用户登录的过程中被使用,当用户登录之后,用户将其id发送给服务器,服务器返回用户当前的状态信息,包括队伍中有哪些人,当前比赛状态,队伍排名等信息。

注:若比赛未开始,则rank为0。

				\paragraph*{Request}
\begin{lstlisting}
GET /api/client/info/id={id}

Host: localhost:8080
Auth:
Content-type: application/json
Accept: application/json
\end{lstlisting}
				\paragraph*{Returns}
>>>>>>> 20eb624d3e96976ee93450dcd011364bb4cf34ae
\begin{lstlisting}
HTTP 200 OK

{
   "memberList":
    [
        "member1":"wangmz",
    	"member2":"songsh"
    ],
    "username": "team1",
    "id":3,
    "rank": 1,

    "gameStatus": "auction"("not_start", "auction_not_record", "auction_recorded", "trade", "rest", "end")
    "round": 0/1/2/3   (第(round+1)轮)
    "timeLeft":300(s)
}
\end{lstlisting}
				\paragraph*{Error}
\begin{lstlisting}
HTTP 404 NOT FOUND
\end{lstlisting}

			\subsubsection{Get Property}
				输入ID,获得与这一ID相关的用户的财产信息。包括机器的使用情况和材料的价格。
				\paragraph*{Request}
\begin{lstlisting}
GET /api/client/property/id={id}

Host: localhost:8080
Auth:
Content-type: application/json
Accept: application/json
\end{lstlisting}

				\paragraph*{Returns}
\begin{lstlisting}
HTTP 200 OK

{
   "wealth": 3000,
   "machine":
    [
        {
            "id": 0073,
            "type": "Cement",
            "left": 3
            "lock":false
        },
        {
            "id": 0793,
            "type": "Brick",
            "left": 0
            "lock":true(正处于出售中的机器和材料 lock == true)
        },
        {
            "id": 8765,
            "type": "Wood",
            "left": 2
            "lock":false
        }
    ],
    "material":
    [
        {
            "type": "Wood",
            "price": 10,
            "number": 20,
            "lock":true
        },
        {
           "type": "Brick",
            "price": 20,
            "number": 0,
            "lock":false
        },
        {
          "type": "Cement",
            "price": 80,
            "number": 150,
            "lock":false
        }
    ]
}
\end{lstlisting}

				\paragraph*{Error}
\begin{lstlisting}
HTTP 404 NOT FOUND
\end{lstlisting}

			\subsubsection{Get All User}
				get 所有队伍,(除了发送消息的队伍),用来发sell Request时进行选择
					\paragraph*{Request}
\begin{lstlisting}
GET /api/client/getAllUser/id={id}

Host: localhost:8080
Auth:
Content-type: application/json
Accept: application/json

\end{lstlisting}
					\paragraph*{Returns}
\begin{lstlisting}
HTTP 200 OK

{
    [
        {
        "teamId": 889,
            "username": "dddd",
        },
        {
               "teamId": 999,
            "username": "ddfadf",
        },
    ]
}
\end{lstlisting}
					\paragraph*{Error}
\begin{lstlisting}
HTTP 404 NOT FOUND
\end{lstlisting}

				\subsubsection{Get Trade History}
					交易历史信息。在发订单的时候客户端手动更新History。
					\paragraph*{Request}
\begin{lstlisting}
GET /api/client/tradehHistory/id={id}

Host: localhost:8080
Auth:
Content-type: application/json
Accept: application/json
\end{lstlisting}
					\paragraph*{Returns}
\begin{lstlisting}
HTTP 200 OK
        [
            {
            "time":   "yyyy-MM-dd'T'HH:mm:ss.SSS'Z'",
            "target": "team1",
            "action": "sell",
            "content": "wood",
            "price": 10,
            "number": 2
            "status": 1,(完成)
            "tradeId":44,
            "buyerId":9
            },

            {
            "time":   "yyyy-MM-dd'T'HH:mm:ss.SSS'Z'",
            "target": "team1",
             "action": "buy",
            "content": "1234" (machine.id ==1234)
            "price": 10,
            "number": 1     (只能是1)
            "status": 0,    (正在进行)
            "tradeId":44,
            "buyerId":9
            },

            {
            "time":   "yyyy-MM-dd'T'HH:mm:ss.SSS'Z'",
            "target": "team1",
            "action": "buy",
            "content": "6666"    (machine.id ==1234)
            "price": 10,
            "number": 1     (只能是1)
            "status": -1,   (失败)
            "tradeId":44,
            "buyerId":9,
            }
        ]\end{lstlisting}
					\paragraph*{Error}
\begin{lstlisting}
HTTP 404 NOT FOUND
\end{lstlisting}

				\subsubsection{Get Produce History}
					生产历史信息。
					\paragraph*{Request}
\begin{lstlisting}
GET /api/client/produceHistory/id={id}

Host: localhost:8080
Auth:
Content-type: application/json
Accept: application/json
\end{lstlisting}
					\paragraph*{Returns}
\begin{lstlisting}
HTTP 200 OK
        [
            {
            "time":   "yyyy-MM-dd'T'HH:mm:ss.SSS'Z'",
            "machineId":9987
            "content": "Brick",
            "price": 10,
            "number": 2
            },
            {
            "time":   "yyyy-MM-dd'T'HH:mm:ss.SSS'Z'",
            "machineId":3457
            "content": "Wood",
            "price": 10,
            "number": 2
            },
            {
            "time":   "yyyy-MM-dd'T'HH:mm:ss.SSS'Z'",
            "machineId":5777
            "content": "Cement",
            "price": 10,
            "number": 2
            },
        ]\end{lstlisting}
					\paragraph*{Error}
\begin{lstlisting}
HTTP 404 NOT FOUND
\end{lstlisting}

		然后是websocket的接口。

				\subsubsection{交易:发出出售请求,buyer监听}
\begin{lstlisting}
 MessageMapping: /api/client/property/sellerId={sellerId}/buyerId={buyerId}

 SendTo: /api/client/property/buyerId={buyerId}

 EndPoint:  http://127.0.0.1:8090/trade

 Get JSon Pattern:(卖方发送)
 {
    "tradeId": '',
    "sellerId":sellerId,
     "buyerId":buyerId,
     "buyer":"TOMATO"
     "typeOrMachineID":"9987"  ("Wood" "Cement" "Brick" OR machineID)
     "price":300,
     "number":7
     "seller":"Rua"
}
Send JSon Pattern:(发给买方)
 {
    "tradeId": tradeId,
    "sellerId": sellerId,
     "buyerId": buyerId,
     "buyer": "TOMATO"
     "typeOrMachineID": "9987"  ("Wood" "Cement" "Brick" OR machineID)
     "price": 300,
     "number": 7
     "seller": "Rua"
 }

\end{lstlisting}
				\subsubsection{交易结束 给卖家转发账单和现有资产}
\begin{lstlisting}
 MessageMapping: /api/client/tradeFinish/sellerId={sellerId}

 SendTo: /api/client/tradeFinish/id={sellerId}

 EndPoint:  http://127.0.0.1:8090/tradeFinish

 Get JSon Pattern:
 {
    "tradeId":tradeId,
    "sellerId":sellerId,
    "buyerId":buyerId,
    "buyer":"TOMATO"
    "typeOrMachineID":"9987"  ("Wood" "Cement" "Brick" OR machineID)
    "price":300,
    "number":7
    "seller":"Rua"

     "isAccept":true
}

Send JSon Pattern:
{
    "reply":
    {
    "tradeId":tradeId,
    "sellerId":sellerId,
    "buyerId":buyerId,
    "buyer":"TOMATO"
    "typeOrMachineID":"9987"  ("Wood" "Cement" "Brick" OR machineID)
    "price":300,
    "number":7
    "seller":"Rua"

    "isAccept":true
     }


   "propertyList":
   {
   	"wealth":1000,
   	"machineList":
    	[
        {
            "id": 0073,
            "type": "Wood",
            "left": 3
            "lock":false
        },
        ],
    	"materialList":
    	[
        {
            "type": "Brick",
            "price": 20,
            "number": 0,
            "lock":false
        },
    	]
    }
}
\end{lstlisting}

				\subsubsection{交易结束 给队友转发账单和现有资产}
\begin{lstlisting}
 MessageMapping: /api/client/tradeFinish/buyerId={buyerId}

 SendTo: /api/client/tradeFinish/id={buyerId}

 EndPoint:  http://127.0.0.1:8090/tradeFinish

 Get JSon Pattern:
 {
    "tradeId":tradeId,
    "sellerId":sellerId,
    "buyerId":buyerId,
    "buyer":"TOMATO"
    "typeOrMachineID":"9987"  ("Wood" "Cement" "Brick" OR machineID)
    "price":300,
    "number":7
    "seller":"Rua"

     "isAccept":true
}

Send JSon Pattern:
{
    "reply":
    {
    "tradeId":tradeId,
    "sellerId":sellerId,
    "buyerId":buyerId,
    "buyer":"TOMATO"
    "typeOrMachineID":"9987"  ("Wood" "Cement" "Brick" OR machineID)
    "price":300,
    "number":7
    "seller":"Rua"

    "isAccept":true
     }


   "propertyList":
   {
   	"wealth":1000,
   	"machineList":
    	[
        {
            "id": 0073,
            "type": "Wood",
            "left": 3
            "lock":false
        },
    	],
    	"materialList":
    	[
        {
            "type": "Brick",
            "price": 20,
            "number": 0,
            "lock":false
        },
    	]
    }
}
\end{lstlisting}

				\subsubsection{监听比赛状态改变}
\begin{lstlisting}
 MessageMapping: /api/client/competition/status/id={id}

 SendTo: /api/client/competitionStatus/id={id}

EndPoint:  http://127.0.0.1:8090/hhh

Send JSon Pattern:
 {
    	"gameStatus": "auction"("not_start", "auction_not_record", "auction_recorded", "trade", "rest", "end")
        "round": 0/1/2/3   (第(round+1)轮)
        "timeLeft":227(s)

 }

 Get JSON Pattern:
 {
    "gameStatus": "auction"("not_start", "auction_not_record", "auction_recorded", "trade", "rest",
    "round": 0/1/2/3   (从0开始)
 }
\end{lstlisting}

				\subsubsection{监听produce后资产的改变}
\begin{lstlisting}
 MessageMapping: /api/client/ListenProperty/id=3

 SendTo: /api/client/ListenProperty/receive/id=3

EndPoint:  http://127.0.0.1:8090/listenProperty

Send JSon Pattern:
{
   "wealth":1000,
   "machine":
    [
        {
            "id": 0073,
            "type": "Wood",
            "left": 3
            "lock":false
        },
        {
            "id": 0793,
            "type": "Brick",
            "left": 0
            "lock":true
        },
        {
            "id": 8765,
            "type": "Cement",
            "left": 2
            "lock":false
        }
    ],
    "material":
    [
        {
           "type": "Wood",
            "price": 10,
            "number": 20,
            "lock":false
        },
        {
           "type": "Brick",
            "price": 20,
            "number": 0,
            "lock":false
        },
        {
           "type": "Cement",
            "price": 80,
            "number": 150,
            "lock":false
        }
    ]
}

Get JSON Pattern:
{
   "id":2,
   "times":1,
}
}
\end{lstlisting}

				\subsubsection{撤销,卖方监听}
\begin{lstlisting}
 MessageMapping: /api/client/undo/sendToSeller/sellerId={sellerId}/buyerId={buyerId}

 SendTo: /api/client/receiveUndo/id={sellerId}

 EndPoint:  http://127.0.0.1:8090/undo

 Get JSon Pattern:(卖方发送)
 {
    "tradeId": 77,
 }
Send JSon Pattern:(发给卖方)
 {
    "request":
    {
    "tradeId":77,
    "sellerId":sellerId,
    "buyerId":buyerId,
    "buyer":"TOMATO"
    "typeOrMachineID":"9987"  ("Wood" "Cement" "Brick" OR machineID)
    "price":300,
    "number":7
    "seller":"Rua"
     }

   "propertyList":
   {
   	"wealth":1000,
   	"machineList":
    	[
        {
            "id": 0073,
            "type": "Wood",
            "left": 3
            "lock":false
        },
    	],
    	"materialList":
    	[
        {
            "type": "Brick",
            "price": 20,
            "number": 0,
            "lock":false
        },
    	]
    }
 }

\end{lstlisting}

				\subsubsection{撤销,买方监听}
\begin{lstlisting}
 MessageMapping: /api/client/undo/sendToBuyer/sellerId={sellerId}/buyerId={buyerId}

 SendTo: /api/client/receiveUndo/id={buyerId}

 EndPoint:  http://127.0.0.1:8090/undo

 Get JSon Pattern:(卖方发送)
 {
    "tradeId": 77,
 }
Send JSon Pattern:(发给买方)
 {
    "request":
    {
    "tradeId":77,
    "sellerId":sellerId,
    "buyerId":buyerId,
    "buyer":"TOMATO"
    "typeOrMachineID":"9987"  ("Wood" "Cement" "Brick" OR machineID)
    "price":300,
    "number":7
    "seller":"Rua"
     }

   "propertyList":
   {
   	"wealth":1000,
   	"machineList":
    	[
        {
            "id": 0073,
            "type": "Wood",
            "left": 3
            "lock":false
        },
    	],
    	"materialList":
    	[
        {
            "type": "Brick",
            "price": 20,
            "number": 0,
            "lock":false
        },
    	]
    }
 }

\end{lstlisting}

		\subsection{需求的可追踪性}

		\subsection{注解}
		
\section{数据库(顶层)设计说明}
\label{数据库(顶层)设计说明}


	\section{软件测试计划}
		\label{软件测试计划}
		\subsection{引言}
			对于后端我们做了详尽的测试,包括Controller正确性的测试,Service的测试和底层数据库的测试。
			\subsubsection{系统概述}
				整个test文件包含三种test,分别是Controller的test,Model的test和Service的test。Controller中的test又分为Admin的test和Client的test。由于没有找到微信websocket的test方法,所以微信的websocket没有写test。
			\subsubsection{文档概述}
				本文档将介绍测试环境,测试计划以及测试进度。
			\subsubsection{与其他计划的关系}
				测试是在接口实现中逐步完善的。与接口的计划相辅相成,可以说是写一个接口就写了一个test。
		\subsection{软件测试环境}
			\subsubsection{后端测试}
				\paragraph{软件项}
					我们所用的语言是Java,使用的框架是Spring Boot,所有的测试都是搭建在这个框架之上的,写测试的方法也是使用了这个框架集成的test annotations。
				\paragraph{参与组织}
					Tomato后端开发组。
				\paragraph{人员}
					张慧盟、宋世虹。
				\paragraph{要执行的测试}
					如上所述,需要测试Controller,Model和Service。
		\subsection{计划}
			\subsection{总体设计}
				\subsubsection{测试过程}
					\subsubsection{Controller}
						\paragraph{Admin}
							\begin{itemize}
							\item AdminLoginControllerTest: 测试了两种情况,Admin登陆成功和登录失败。
							\item AuctionControllerTest: 测试了Auction的接口,包括当get auction的时候应该返回什么,如果发起了错误的get auction操作时会返回什么,post auction的时候Service的更新, post auction错误的时候返回什么等等。
							\item ChangePasswordControllerTest: 测试了如果password输入正确的更新准则和输入错误时的报错。
							\item CompetitionCreatorControllerTest:测试了创建比赛,更新比赛和获得比赛具体信息时的返回值,以及上述情况request出错时的返回值。
							\item CompetitionInfomationControllerTest:测试了请求比赛信息时返回信息的正确性,和请求失败的返回值。
							\item CompetitionStatusAdminControllerTest:测试了不同比赛状态时的返回值。
							\item DeleteCompetitionControllerTest:测试了删除比赛时:正常删除、没有比赛和获得全部比赛出错的情况。
							\item GetAllCompetitionControllerTest:测试了返回全部比赛时:正常返回、没有比赛、获得全部比赛出错的情况。
							\item UpdateCompetitionStatusControllerTest:测试了更新比赛状态时的返回值:正常返回、没有比赛、后端没法update比赛状态、非法状态等情况。
							\end{itemize}
						\paragraph{Client}
							\begin{itemize}
							\item BuyMaterialControllerTest:测试了买东西时:买材料和买机器的正在交易、已经交易完成和交易取消的情况。
							\item ClientInfoControllerTest:测试了获取和更新team信息时成功和失败时的情况。
							\item ClientLoginControllerTest:测试了登陆成功和失败的情况。
							\item ClientPropertyControllerTest:测试了team获取property时成功、失败的情况,测试了team生产的时候成功和失败的情况,
							\item CompetitionStatusControllerTest:测试了更新比赛状态时Client的返回情况。
							\item GetAllUsersControllerTest:测试了获取所有team的成功和失败的情况。
							\item GetProduceHistoryControllerTest:测试了获取生产历史的成功和失败(没有team,没有produce记录)的情况。
							\item GetTradeHistoryControllerTest:测试了获取交易历史的成功和失败(没有team,没有produce记录)的情况。
							\item ListenPropertyControllerTest:测试了在websocket转发中监听produce后资产的改变。
							\item SellMaterialControllerTest:测试了卖资产时成功的情况,包括各种材料和机器。
							\item SendToSellerControllerTest:测试了将售卖的账单转给买方和卖方的同team,主要测试了不同账单状态的回复情况。
							\item UndoTradeControllerTest:测试了撤销交易的成功情况,主要测试了发给买方和卖方的账单。
							\end{itemize}
					\subsubsection{Service}
						\begin{itemize}
							\item admin.CompetitionServiceTest: 测试了创建比赛、查找比赛、更新比赛、删除比赛的接口,测试了创建队伍、更新队伍、查找队伍的接口,测试了创建机器、查找机器、更新机器的接口,测试了获得所有Produce信息和交易信息的接口,测试了更新Round的接口。
							\item admin.LoginServiceTest: 测试了创建Admin、更新Admin、查找Admin和删除Admin的接口。
							\item user.CompetitionServiceTest: 测试了查找Competition和更新Competition的成功和失败的情况。
							\item user.LoginServiceTest: 测试了查找team、更新team、创建team和删除team的Service的成功和失败情况。
							\item user.OrderServiceTest: 测试了创建Produce、查找Produce、更新Produce、删除Produce的接口,测试了创建Trade、 查找Trade、更新Trade、删除Trade的接口。
						\end{itemize}
					\subsubsection{Model}
						\begin{itemize}
							\item AdminTest: 测试了Admin表的各field。
							\item CompetitionTest: 测试了Competition表的各field,包括Round。
							\item MachineTest: 测试了Machine表的各field。
							\item MaterialTest: 测试了Material的struct是否正确。
							\item ProduceTest: 测试了Produce表的各field。
							\item TeamTest: 测试了Team表的各field。
							\item TradeTest: 测试了Trade表的各field。
						\end{itemize}
			\subsection{计划执行的测试}
				\subsubsection{被测试项}
					Tomato的整个后端。
		\subsection{测试进度表}
			如下图所示。

			\includegraphics[scale = .3]{fig/test_total_time.jpeg}
		\subsection{需求的可追踪性}
			本测试计划覆盖了所有后端(除微信WebSocket外)的需求(接口)。
		\subsection{评价}
			\subsubsection{评价准则}
				Coverage。
			\subsubsection{数据处理}
				数据处理由IntelliJ自动完成,IntelliJ将为我们生成完整的Coverage报告。
			\subsubsection{结论}
				本测试计划十分合理。
		\subsection{注解}
	% \section{软件测试说明}
	% 	%unfinished 不过感觉和测试计划东西一样???
	% 	\subsection{引言}
	% 		\subsubsection{标识}
	% 		\subsubsection{系统概述}
	% 		\subsubsection{文档概述}
	% 	\subsection{引用文件}
	% 	\subsection{测试准备}
	% 	\subsection{测试说明}
	% 	\subsection{需求的可追踪性}
	\section{软件测试报告}
		\subsection{引言}
			\subsubsection{系统概述}
				可以参见《\ref{软件测试计划}》中的系统概述。
			\subsubsection{文档概述}
				本文档将介绍测试的详细结果,主要是Coverage。
		\subsection{测试结果概述}
			\subsubsection{对被测试软件的总体评估}
				\includegraphics[scale = .3]{fig/test_1.jpg}

				本测试覆盖率很高,可以说是十分到位了。
			\subsubsection{测试环境的影响}
				任何平台都可以进行测试,并没有什么影响。
			\subsubsection{改进建议}
				如果可以的话,可以绕出架构对微信的websocket进行测试。
		\subsection{详细的测试结果}
			以下是我们的测试覆盖率:

			\includegraphics[scale = .3]{fig/test_1.jpg}
		
			\includegraphics[scale = .3]{fig/test_2.jpg}
		
			\includegraphics[scale = .3]{fig/test_3.jpg}
		
			\includegraphics[scale = .3]{fig/test_4.jpg}
		
			\includegraphics[scale = .3]{fig/test_5.jpg}
		
			\includegraphics[scale = .3]{fig/test_6.jpg}
		
			\includegraphics[scale = .3]{fig/test_7.jpg}
		
			\includegraphics[scale = .3]{fig/test_8.jpg}
		
			\includegraphics[scale = .3]{fig/test_9.jpg}
		\subsection{测试记录}
			本次测试结果生成于2018年1月19日,测试平台是Mac,集成开发环境是IntelliJ。见证者是宋世虹和张慧盟。
		\subsection{评价}
			\subsubsection{能力}
				该测试覆盖了几乎所有后台的接口,可以说是很鲁棒,给前端提供了十分稳定的服务。
			\subsubsection{缺陷和限制}
				由于没有找到微信的websocket的测试方法,所以我们没有对微信的websocket进行测试,这很遗憾。同时,由于时间的关系,还有极少一部分代码没有被测试到。
		\subsection{测试活动总结}
			人力消耗:张慧盟、宋世虹。


\end{document}
